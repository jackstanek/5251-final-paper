\documentclass[mathserif,notheorems]{beamer} % option notheorems
\usepackage{amssymb, amsmath, amsthm, amsfonts}
\usepackage{cleveref}
\usepackage{thmtools}
\usetheme{Berlin}
\title{A Review of Lossless Data Compression Algorithms}
\author{Jack Stanek}

\declaretheorem[numberwithin=section]{Theorem}

\begin{document}
\begin{frame}
  \titlepage
\end{frame}

\begin{frame}
  \tableofcontents
\end{frame}

\begin{frame}
  \section{A Brief Introduction to Data Compression}
  \subsection{Informal Overview}
  \frametitle{What is data compression?}

  Informally, a data compression scheme is some method for reducing
  the amount of space required to store a given amount of data.

  Some commonplace examples:

  \begin{itemize}
  \item \texttt{*.mp3} files
  \item \texttt{*.zip} files
  \item Abbreviations in writing
  \end{itemize}

  This brief list includes some lossy methods, though this
  presentation will cover only lossless methods.
\end{frame}

\begin{frame}
  \subsection{Theoretical Background}
  Information can be represented as a series of symbols
  \begin{itemize}
  \item Text is a string of letters in langauge's alphabet
  \item Numbers are strings of Arabic numerals
  \end{itemize}
  For some alphabet $\Sigma$, the set of all compression schemes
  $C_\Sigma$ is the set of bijections
  $c : \Sigma^n \to \bigcup_{k = 1}^{n} \Sigma^k$ for all
  $n \in \mathbb{N}$ such that $n \geq 2$.
\end{frame}

\begin{frame}
  \subsection{Impossibility of perfect compression}
  A compression scheme is considered ``perfect'' if and only if its
  image is the set of all strings of length strictly less than that of
  the domain.
  \begin{Theorem}
    No perfect compression scheme exists.
  \end{Theorem}
\end{frame}

\begin{frame}
  \subsection{Maximum Compressibility of Data}
  \begin{itemize}
  \item This result suggests there is a limit to how small compressed
    data can be.
  \item Shannon's Noiseless Coding gives a lower bound.
    \begin{itemize}
    \item The average word size of a prefix code on a channel is
      greater than or equal to its entropy.
    \item Since human data sources are typically very regular
      (i.e. have low information entropy), data compression methods
      are able to greatly reduce the sizes of these sorts of data.
    \end{itemize}
  \end{itemize}
\end{frame}

\begin{frame}
  \section{Lempel-Ziv Based Algorithms}
  \subsection{LZ}
  \frametitle{Lempel-Ziv compression algorithms}
  LZ77 and LZ78 (a.k.a. LZ1 and LZ2, a.k.a. LZ)
  \begin{itemize}
  \item Introduced in 1977 and 1978 by Abraham Lempel and Jakob Ziv
  \item Dictionary block compressor family
  \item Uses a sliding-window approach
  \item Produces 3-tuples representing prefixes
  \end{itemize}
\end{frame}

\begin{frame}
  \subsection{LZ77 Example}
  Suppose we are to encode the tongue-twister ``Can you can a can as a
  canner can can a can?'' with a window size of 8.
\end{frame}

\begin{frame}
  \section{FLAC}
  \frametitle{Free Lossless Audio Codec}
  The Free Lossless Audio Codec is a domain-specific compression
  scheme for hi-fi audio.
\end{frame}

\end{document}

\documentclass[mathserif,notheorems]{beamer} % option notheorems
\usepackage{amssymb, amsmath, amsthm, amsfonts}
\usepackage{cleveref}
\usepackage{thmtools}
\usetheme{Berlin}
\title{A Review of Lossless Data Compression Algorithms}
\author{Jack Stanek}

\declaretheorem[numberwithin=section]{Theorem}

\begin{document}
\begin{frame}
  \titlepage
\end{frame}

\begin{frame}
  \tableofcontents
\end{frame}

\begin{frame}
  \section{A Brief Introduction to Data Compression}
  \subsection{Informal Overview}
  \frametitle{What is data compression?}

  A data compression scheme refers to any system for reducing the
  number of symbols required to represent arbitrary information.

  Some commonplace (possibly lossy!) examples:

  \begin{itemize}
    \item \texttt{*.mp3} files
    \item \texttt{*.zip} files
    \item Abbreviations in writing
  \end{itemize}

  Clearly, this list is not exhaustive, but it provides an intuition
  to how these systems function.
\end{frame}

\begin{frame}
  \subsection{Theoretical Background}
  Information can be represented as a series of symbols
  \begin{itemize}
    \item Text is a string of letters in langauge's alphabet
    \item Numbers are strings of Arabic numerals
  \end{itemize}
  For some alphabet $\Sigma$, the set of all compression schemes
  $C_\Sigma$ is the set of bijections
  $c : \Sigma^n \to \bigcup_{k = 1}^{n} \Sigma^k$ for all
  $n \in \mathbb{N}$ such that $n \geq 2$. A compression scheme is
  considered ``perfect'' if and only if its image is the set of all
  strings of length strictly less than that of the domain.
\end{frame}

\begin{frame}
  \subsection{Impossibility of perfect compression}
  \begin{Theorem}
    No perfect compression scheme exists.
  \end{Theorem}
\end{frame}

\begin{frame}

\end{frame}

\end{document}
